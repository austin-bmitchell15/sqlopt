\documentclass[sigconf]{acmart}

\setcopyright{none}    % Disables ACM copyright notice
\acmConference{}{}{}   % Removes conference information
\acmBooktitle{}        % Removes book title
\acmDOI{}              % Removes DOI
\acmISBN{}             % Removes ISBN
\settopmatter{printacmref=false} % Removes the ACM reference format notice
\renewcommand\footnotetextcopyrightpermission[1]{}


\title{Project Proposal: SQL Auto-Optimization Using Random Forests}
\author{Austin Mitchell}
\affiliation{%
  \institution{Georgia Institute of Technology}
  \city{Atlanta, GA}
  \country{United States}
}
\email{amitchell80@gatech.edu}

\begin{document}

\maketitle

\section{Introduction}
Query optimization still remains an issue that has yet to be solved.
Traditional optimization techniques use heuristics and cost-based optimization technniques.
These techniques typically work by analyzing the execution costs and determining more efficient ways to establish these queries.
This can be done through the manipulation of operations such as reordering join operations or choosing the right indexes. These optimizations
are important for a scalable database to be able to handle queries that require lots of computational power on the system hosting these services.
While these optimizations are good by utilizing heuristics, they may not always find the best solutions.
By improving these optimizations, this could lead to decreased costs and more efficient query times which could benefit both the user
and the producer. As is the case with most problems, there are plenty of researchers who seek to expand upon this idea and improve these
optimizers further.

Machine learning introduces an alternative by ``teaching'' itself from past query executions to and suggest optimizations.
Several studies have looked into this idea of developing optimizers to replace current standards\cite{Marcus_2019}\cite{mlcomparative2024}.
This project attempts to build upon this initial idea by working to develop a machine learning optimizer that works in conjuction with the
traditional cost-based optimizers that currently exist within database management systems (DBMS).

This machine learning optimizer will attempt to predict execution times and suggest rewrites with the hopes that it can improve performance
of query executions.

\section{Methodology}

\subsection{TPC-H Benchmark}
In order to develop this project, we will be using the TPC-H dataset in order to generate real-world queries. This is a widely-used performance
benchmark, and thus it should work for this scenario. Additionally, since the TPC-H dataset works as a benchmark, it could prove useful to compare my
work to those of others in order to develop a comparative understanding to the project as a whole. We will execute the benchmark and 
use the \texttt{EXPLAIN ANALYZE} command in PostGres we will collect each queries various features to be used in training a RandomForest model
in order to predict execution times of the query. PostGres was chosen almost solely for its ability to provide detailed plans and logging of these queries.
These results can then be stored in some lightweight way such as a CSV file.

\subsection{Model Suggestion and Prediction}
From this CSV file, important features that would be important to the execution time can be extracted. We will utilize a number of features including but not limited to:
number of joins, indexes used, sorting/aggregation operations, etc. These features will then be analyzed by a random forest model
in order to determine which features are the most important and provide a predicted execution time. The random forest will utilized the actual
execution time as its validation information in an attempt to improve its own accuracy. This outlines my 75\% goal for this project which is to
obtain a model that is able to predict the execution time of a given query to some degree.

The next step of this project would then come in the form of developing an alternative execution plan analysis. Currently, I would like to explore the
idea of using Genetic Algorithms alongside the Random Forest. This Genetic Algorithm would attempt to perform rewrites of the original query, and
use our newly created Random Forest model in order to predict its own execution time. However, if I find the Genetic Algorithm to be unfeasible at scale,
I might attempt to utilize a Reinforcement Learning or Neural Network model instead. At a high-level, a genetic algorithm seems suitable for this type of situation; however,
I am wary of their inability to ``learn'' the alternative execution plan creation process. In terms of analysis, this type of algorithm might work, but in real-world applications
it could be too slow. A comparison of this predicted execution time and the execution time of the original query will then be compared to determine which of the population remains.

These two models in conjunction will then be analyzed and compared against the original optimizer of PostGres in order to determine the overall
effectiveness of these two models as a whole. We will perform this analysis across key metrics such as the Random Forest's prediction accuracy and efficiency
improvements of the Genetic Algorithm at a minimum. I will also investigate teh hardware and resource usage of the two systems in an attempt to determine
if the Random Forest and Genetic Algorithm models are feasible at scale.

\section{Resources}
For the required resources, I will be utilizing a PostGreSQL in order to execute the TPC-H benchmark. This TPC-H benchmark will
be used for processing queries that will serve as my real-world test for my own model. I will also be utilizing the Python language and
its numerous machine learning-based libraries in order to develop the models required to improve the optimizer.

\section{Conclusion}
For this project, I hope to achieve goals that surpass the existing Cost-Based Optimizers that exist on most current DBMS's, but
I also understand that Machine Learning models can be computationally expensive especially when they require numerous iterations. My initial goal
is to establish a execution time prediction model. After which, I will attempt to develop a Genetic Algorithm that will attempt to derive alternative
plans for each query and analyze them. I also hope to develop a top-down analysis of both the model and the original PostGreSQL optimizer in hopes
of developing a more complete understanding.


\bibliographystyle{ACM-Reference-Format}
\bibliography{references}

\end{document}

